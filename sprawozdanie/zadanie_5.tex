\section{Przypadek Testowy 5 - Tabu Search - Test statystyczny}
  \subsection{Cel:}
    W teście statystycznym porównamy działanie algorytmu Tabu Search oraz wcześniej zaimplementowanego algorytmu 2-opt.
    Hipotezą startową jest stwierdzenie, iż Tabu Search średnio zwraca lepsze wyniki niż algorytm 2-opt.
    Do analizy uzyskanych wyników zostanie wykorzystany test Wilcoxon'a dla par obserwacji.
  \subsection{Założenia:}
    Do wykonania tego testu wykorzystano instancje znajdujące się w bibliotece \textit{TSPLIB}. Dla Tabu Search, jak i dla 2-opt, początkowe permutacje zostały wygenerowane losowo.
  \subsection{Wykorzystane instancje: }
    W tym teście zostały wykorzystane następujące instancje z biblioteki \textit{TSPLIB}:
    \begin{enumerate}
      \item berlin52.tsp
      \item bier127.tsp
      \item ch150.tsp 
      \item eil76.tsp
      \item gr120.tsp 
      \item gr48.tsp
      \item hk48.tsp
      \item kroA150.tsp
      \item kroB100.tsp
      \item kroC100.tsp
      \item kroD100.tsp
      \item pr107.tsp
      \item pr124.tsp
      \item pr144.tsp
      \item pr76.tsp
      \item st70.tsp
      \item u159.tsp
    \end{enumerate}

    \subsection{Test statystyczny Wilcoxona: }
    \textbf{Hipoteza zerowa: $TABU >= OPT$}, gdzie OPT oznacza algorytm 2-opt, a Tabu- algorytm Tabu Search. \\
    \textbf{Hipoteza alternatywna: $TABU < OPT$ }
    \begin{table}[H]
    \begin{tabular}{|c | c | c | c | c | c |} 
     \hline
     Pair & TABU & OPT & Abs.Diff & Rank & Sign \\ [0.5ex] 
     \hline\hline
      16 & 693 & 698 & 5 & 1 & -1 \\
      4 & 563 & 575 & 12 & 2 & -1 \\
      5 & 7440 & 7463 & 23 & 3 & -1 \\
      11 & 22185 & 22292 & 107 & 4 & -1 \\
      10 & 22110 & 21760 & 350 & 5 & +1 \\
      1 & 8201 & 7829 & 372 & 6 & +1 \\
      6 & 5613 & 5208 & 405 & 7 & +1 \\
      3 & 7505 & 7042 & 463 & 8 & +1 \\
      7 & 12423 & 11941 & 482 & 9 & +1 \\
      8 & 30436 & 29681 & 755 & 10 & +1 \\
      9 & 24676 & 23556 & 1110 & 11 & +1 \\
      12 & 46033 & 48564 & 2531 & 12 & -1 \\
      15 & 115958 & 118846 & 2888 & 13 & -1 \\ 
      17 & 48240 & 44887 & 3353 & 14 & +1 \\
      14 & 63132 & 59638 & 3494 & 15 & +1 \\
      2 & 130696 & 125647 & 5049 & 16 & +1 \\
      13 & 69337 & 64270 & 5067 & 17 & +1 \\
     \hline
    \end{tabular}
    \caption{Tabela rang dla testu Wilcoxona}
    \end{table}

    $W_{-} = 35$.\\
    $W_{+} = 118$. \\
    Wartość mniejsza to $W_{-}$, więc $T=35$.

  \subsection{Wnioski: }
  Wartość krytyczna $\alpha = 0.05$. Dla tego typu statystyki $T_{crit}=41$ (dana z tabeli dla hipotez "o jednym ogonie"). Hipoteza zerowa jest odrzucona, gdy $ T \leq 41 $. U nas $T=35$, jako minimum z $W_{-}$ i $W_{+}$, więc hipoteza zerowa zostaje odrzucona.