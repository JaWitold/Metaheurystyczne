\section{Przypadek Testowy 6 - Test statystyczny - Porównanie działania GA z 2OPT}
  \subsection{Cel:}
    W teście statystycznym porównanym działanie algorytmu genetycznego z algorytmem TABU,
    którego implementacją zajęliśmy się w ramach listy 2. Hipotezą startową jest stwierdzenie, że
    algorytm genetyczny zwraca lepszą średnią medianę wyników niż algorytm TABU. Do analizy
    uzyskanych wyników zostanie wykorzystany test Wilcoxon'a dla par obserwacji (one-sided).
  \subsection{Założenia: }
    Do wykonania tego testu wykorzystano instancje znajdujące się w bibliotece TSPLIB. Początkowe permutacje są generowane w sposób
    losowy.
  \subsection{Wykorzystane instancje: }
    W tym teście zostały wykorzystane następujące instancje z biblioteki TSPLIB:
    \begin{enumerate}
      \item berlin52.tsp
      \item bier127.tsp
      \item ch150.tsp
      \item eil76.tsp
      \item gr120.tsp
      \item gr48.tsp
      \item hk48.tsp
      \item kroA150.tsp
      \item kroB100.tsp
      \item kroC100.tsp
      \item kroD100.tsp
      \item pr107.tsp
      \item pr124.tsp
      \item pr144.tsp
      \item pr76.tsp
      \item st70.tsp
      \item u159.tsp
    \end{enumerate}
  \subsection{Test: }
    \textbf{Hipoteza zerowa: } GA $>=$ TABU \\
    \textbf{Hipoteza alternatywna: } GA $<$ TABU \\
    \begin{table}[H]
      \centering
      \begin{tabular}{| c | c | c | c | c | c |}
        \hline
        Pair & GA & TABU & Abs.Diff & Rank & Sign \\
        \hline
        6 & 11941 & 11976 & 35 & 1 & -1 \\
        15 & 678 & 718 & 40 & 2 & -1 \\
        5 & 5214 & 5283 & 69 & 3 & -1 \\
        4 & 7703 & 7895 & 192 & 4 & -1 \\
        1 & 8098 & 8295 & 197 & 5 & -1 \\
        8 & 23508 & 23302 & 206 & 6 & +1 \\
        3 & 7964 & 7332 & 632 & 7 & +1 \\
        13 & 64418 & 63677 & 741 & 8 & +1 \\
        9 & 21360 & 22141 & 781 & 9 & -1 \\
        10 & 23717 & 25243 & 1526 & 10 & -1 \\
        11 & 46617 & 48680 & 2063 & 11 & -1 \\
        12 & 69279 & 71421 & 2142 & 12 & -1 \\
        2 & 132632 & 130217 & 2415 & 13 & +1 \\
        7 & 31533 & 28460 & 3073 & 14 & +1 \\
        16 & 53914 & 49123 & 4791 & 15 & +1 \\
        14 & 109056 & 118174 & 9118 & 16 & -1 \\
        \hline
          
      \end{tabular}
      \caption{Tabela rang dla testu Wilcoxona dla wartości funkcji celu}
      $W_{-} = 1+2+3+4+5+9+10+11+12+16=73$ \\
      $W_{+} = 6+7+8+13+14+15 $ \\
      $min(W_{+},W_{-}) = 63 = T$ \\
  
    \end{table}
    Poniżej powtórzony test dla wartości PRD: \\
    \textbf{UWAGA:} W kolumnach \textit{GA} oraz \textit{TABU} wartości podawane są w \textbf{PROCENTACH!}. Ze względów estetycznych w samych tabelach oznaczenia procenta (\%) zostały pominięte.
    \begin{table}[H]
      \centering
      \begin{tabular}{| c | c | c | c | c | c |}
        \hline
        Pair & GA & TABU & Abs.Diff & Rank & Sign \\
        \hline
        6 & 4.2 & 4.5 & 0.3 & 1 & -1 \\
        8 & 6.1 & 5.2 & 0.9 & 2 & +1 \\
        13 & 10 & 8.8 & 1.2 & 3 & +1 \\
        5 & 3.3 & 4.6 & 1.3 & 4 & -1 \\
        2 & 12 & 10 & 2 & 5 & +1 \\
        1 & 7 & 10 & 3 & 6.5 & -1 \\
        4 & 11 & 14 & 3 & 6.5 & -1 \\
        12 & 17.3 & 21 & 3.7 & 8 & -1 \\
        9 & 2.9 & 6.7 & 3.8 & 9 & -1 \\
        11 & 5.2 & 9/9 & 4.7 & 10 & -1 \\
        15 & 0.1 & 6.3 & 6.2 & 11 & -1 \\
        10 & 11.3 & 18.6 & 7.3 & 12 & -1 \\
        14 & 0.8 & 9.3 & 8.5 & 13 & -1 \\
        3 & 22 & 12 & 10 & 14 & +1 \\
        16 & 28.1 & 16.7 & 11.4 & 15 & +1 \\
        7 & 18.8 & 7.3 & 11.5 & 16 & +1 \\
        \hline
          
      \end{tabular}
      \caption{Tabela rang dla testu Wilcoxona dla wartości PRD}
      $W_{-} = 1+4+6.5+6.5+8+9+10+11+12+13=81$ \\
      $W_{+} = 2+3+5+14+15+16=55 $ \\
      $min(W_{+},W_{-}) = 55 = T$ \\
  
    \end{table}
    W obu przypadkach wnioski będą identyczne.
  \subsection{Wnioski: }
    Wartość krytyczna $\alpha = 0.05$, dla tego typu statystyk $T_{crit} = 35$ (dana z tabeli dla hipotez "o jednym ogonie").
    Nie możemy odrzucić hipotezy zerowej, gdyż nie ma wystarczających dowodów aby spekulować o tym, iż różnica median w tym przypadku jest mniejsza niż 0 ($ T > 35$).