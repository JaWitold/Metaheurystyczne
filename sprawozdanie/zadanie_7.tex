\section{Przypadek Testowy 7 - Test statystyczny - Porównanie działania GA z 2OPT}
  \subsection{Cele:}
  W teście statystycznym porównanym działanie algorytmu genetycznego z algorytmem 2OPT,
  którego implementacją zajęliśmy się w ramach listy 2. Hipotezą startową jest stwierdzenie, że
  algorytm genetyczny zwraca lepszą średnią medianę wyników niż algorytm 2OPT. Do analizy
  uzyskanych wyników zostanie wykorzystany test Wilcoxon'a dla par obserwacji (one-sided).
  \subsection{Założenia: }
  Do wykonania tego testu wykorzystano instancje znajdujące się w bibliotece TSPLIB. Początkowe permutacje są generowane w sposób
  losowy.
  \subsection{Wykorzystane instancje: }
  Lista wykorzystanych instancji jest identyczna do tej, która znajduje się dla 6 przypadku testowego.
    \subsection{Wnioski: }
    \subsection{Test: }
    \textbf{Hipoteza zerowa: } GA $>=$ TABU \\
    \textbf{Hipoteza alternatywna: } GA $<$ TABU \\
    \begin{table}[!ht]
      \centering
      \begin{tabular}{| c | c | c | c | c | c |}
        \hline
        Pair & GA & 2OPT & Abs.Diff & Rank & Sign \\
        6 & 11941 & 11930 & 11 & 1 & +1 \\
        15 & 678 & 696 & 18 & 2 & -1 \\
        5 & 5214 & 5304 & 90 & 3 & -1 \\
        8 & 23508 & 23325 & 183 & 4 & +1 \\
        4 & 7703 & 7423 & 280 & 5 & +1 \\
        1 & 8098 & 7811 & 287 & 8 & +1 \\
        13 & 64418 & 64025 & 393 & 7 & +1 \\
        11 & 46617 & 47035 & 418 & 8 & -1 \\
        3 & 7964 & 7098 & 866 & 9 & +1 \\
        9 & 21360 & 22364 & 1004 & 10 & -1 \\
        10 & 23717 & 21866 & 1851 & 11 & +1 \\
        7 & 31533 & 28218 & 3315 & 12 & +1 \\
        14 & 109056 & 112479 & 3423 & 13 & =1 \\
        2 & 132632 & 127530 & 5102 & 14 & +1 \\
        12 & 69279 & 64071 & 5208 & 15 & +1 \\
        16 & 53914 & 46294 & 7620 & 16 & +1 \\

        \hline
          
      \end{tabular}
      \caption{Tabela rang dla testu Wilcoxona}
      $W_{-} = 2+3+8+10+13=36 $
      $W_{+} = 1+4+5+6+7+9+11+12+14+15+16 = 100 $
      $min(W_{+},W_{-}) = 36 = T$
  
    \end{table}
  \subsection{Wnioski: }
  Wartość krytyczna $\alpha = 0.05$, dla tego typu statystyk $T_{crit} = 35$ (dana z tabeli dla hipotez "o jednym ogonie").
  Nie możemy odrzucić hipotezy zerowej, gdyż nie ma wystarczających dowodów aby spekulować o tym, iż różnica median w tym przypadku jest mniejsza niż 0 ($ T > 35$).
    